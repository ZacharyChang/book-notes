\documentclass[geye,green,pad,cn]{elegantnote}

\title{【习惯的力量】读书笔记}
\author{Zachary}
\version{0.0.1}
\date{2019 年 6 月 1 日}

\begin{document}
\maketitle
\tableofcontents

\section{前言}
本书主要由三部份构成,第一部分阐述了习惯如何在个人的生活中产生。第二部分研究了成功的公司和组织的习惯,并通过具体案例表述了如何通过习惯改进公司或组织中的效率。第三部分探讨了习惯是如何在社会行为中产生影响的。

本书中心论点在于:如果弄清楚习惯运作的原理及方式,则习惯是可以被调整或改变的。

\section{第一部分 个体的习惯}
\subsection{第一章 习惯是如何运作的}
尤金的事例: 一名脑组织受损的患者也能形成某种固定习惯,即使记忆缺失。

老鼠实验:  将老鼠放入有巧克力的T字形迷宫中,随着重复的实验,老鼠熟悉了路线并形成了急速前进的习惯,同时大脑的思考越来越少。基底核是控制自动行为和存储习惯的组织。

【组件化】: 大脑将一系列行为  变成一种自动的惯常行为,这是习惯形成的基础

\end{document}